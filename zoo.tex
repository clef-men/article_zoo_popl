\section{\Zoo in practice}
\label{sec:zoo}

% In this section, we give an overview of our framework.
%We also provide a minimal example%
%\footnote{
%\url{https://github.com/clef-men/zoo-demo}
%}
%demonstrating its use.

The core of \Zoo is \ZooLang: a concurrent, imperative, untyped, functional programming language fully formalized in \Rocq.
Its semantics has been designed to match \OCaml.

\ZooLang comes with a program logic based on \Iris, proved correct with respect to its small-step operational semantics.
The reasoning rules are expressed in separation logic (including rules for the different constructs of the language) along with \Rocq tactics that integrate into the \Iris proof mode~\citep{DBLP:journals/pacmpl/KrebbersJ0TKTCD18}.
In addition, it supports \Diaframe~\citep{DBLP:conf/pldi/MulderKG22,DBLP:journals/pacmpl/MulderK23}, enabling proof automation.

\subsection{Translation from \OCaml to \ZooLang}

\begin{figure}[tp]
\begin{minipage}[t]{0.42\textwidth}
\begin{ocamlcode}
type 'a t = 'a list Atomic.t
let create () = Atomic.make []

let rec push t v =
  let old = Atomic.get t in
  let new_ = v :: old in
  if not (Atomic.compare_and_set
            t old new_)
  then (
    Domain.cpu_relax () ;
    push t v
  )

let rec pop t =
  match Atomic.get t with
  | [] -> None
  | v :: new_ as old ->
      if Atomic.compare_and_set
           t old new_
      then Some v
      else (
        Domain.cpu_relax () ;
        pop t
      )
\end{ocamlcode}
\end{minipage}
\begin{minipage}[t]{0.49\textwidth}
\begin{coqcode}
Definition stack_create : val :=
  fun: <> => ref [].

Definition stack_push : val :=
  rec: "push" "t" "v" =>
    let: "old" := !"t" in
    let: "new_" := "v" :: "old" in
    if: ~ CAS "t".[contents] "old" "new_"
    then (
      domain_cpu_relax () ;;
      "push" "t" "v"
    ).

Definition stack_pop : val :=
  rec: "pop" "t" =>
    match: !"t" with
    | [] => §None
    | ("v" :: "new_") as "old" =>
        if: CAS "t".[contents] "old" "new_"
        then ‘Some( "v" )
        else (
          domain_yield () ;;
          "pop" "t"
        )
    end.

\end{coqcode}
\end{minipage}

\caption{A concurrent stack in \OCaml and its automatic \ZooLang translation}
\label{fig:stack}
\end{figure}

%%% Local Variables:
%%% mode: LaTeX
%%% TeX-master: "../main"
%%% End:


While \ZooLang lives in \Rocq, we want to verify \OCaml programs.
To connect them we provide the tool \ocamlToZoo to translate \OCaml source files\footnote{Actually, \ocamlToZoo processes binary annotation files (\texttt{.cmt} files).} into \Rocq files containing \ZooLang code.
This tool can process entire \texttt{dune} projects, and supports library dependencies.

The supported \OCaml fragment includes: tuples, variants, records and inline records, shallow \ocamlinline{match}, atomic record fields, unboxed types, toplevel mutually recursive functions.

In \cref{fig:stack} we include the \OCaml implementation of a simple lock-free concurrent stack~\citep{thomas1986systems}, and its automatic translation to \ZooLang, demonstrating that readability is preserved.
Readability is important as users constantly see program fragments during interactive verification.

\subsection{The full language}

\begin{figure}[tp]
\centering
\begin{tabular}{llcl}
    \Rocq term &
    $t$
  \\
    constructor &
    $C$
  \\
    projection &
    $\mathit{proj}$
  \\
    record field &
    $\mathit{fld}$
  \\
    identifier &
    $s, f$
    & $\in$ &
    $\mathrm{String}$
  \\
    integer &
    $n$
    & $\in$ &
    $\mathbb{Z}$
  \\
    boolean &
    $b$
    & $\in$ &
    $\mathbb{B}$
  \\
    binder &
    $x$
    & $\Coloneqq$ &
    $\texttt{<>} \mid s$
  \\
    unary operator &
    $\oplus$
    & $\Coloneqq$ &
    $\texttt{\raisebox{0.5ex}{\texttildelow}} \mid \texttt{-}$
  \\
    binary operator &
    $\otimes$
    & $\Coloneqq$ &
    $\texttt{+} \mid \texttt{-} \mid \texttt{*} \mid \texttt{`quot`} \mid \texttt{`rem`} \mid \texttt{`land`} \mid \texttt{`lor`} \mid \texttt{`lsl`} \mid \texttt{`lsr`}$
  \\
    && | &
    $\texttt{<=} \mid \texttt{<} \mid \texttt{>=} \mid \texttt{>} \mid \texttt{=} \mid \texttt{≠} \mid \texttt{==} \mid \texttt{!=}$
  \\
    && | &
    $\texttt{and} \mid \texttt{or}$
  \\
    expression &
    $e$
    & $\Coloneqq$ &
    $t \mid s \mid \texttt{\#} n \mid \texttt{\#} b$
  \\
    && | &
    $\texttt{fun:}\ x_1 \dots x_n\ \texttt{=>}\ e \mid \texttt{rec:}\ f\ x_1 \dots x_n\ \texttt{=>}\ e \mid e_1\ e_2$
  \\
   && | &
   $\texttt{let:}\ x\ \texttt{:=}\ e_1\ \texttt{in}\ e_2 \mid e_1\ \texttt{;;}\ e_2$
  \\
    && | &
    $\texttt{let:}\ f\ x_1 \dots x_n\ \texttt{:=}\ e_1\ \texttt{in}\ e_2 \mid \texttt{letrec:}\ f\ x_1 \dots x_n\ \texttt{:=}\ e_1\ \texttt{in}\ e_2$
  \\
    && | &
    $\texttt{let:}\ \texttt{‘} C\ x_1 \dots x_n\ \texttt{:=}\ e_1\ \texttt{in}\ e_2 \mid \texttt{let:}\ x_1 \texttt{,} \dots \texttt{,} x_n\ \texttt{:=}\ e_1\ \texttt{in}\ e_2$
  \\
    && | &
    $\oplus e \mid e_1 \otimes e_2$
  \\
    && | &
    $\texttt{if:}\ e_0\ \texttt{then}\ e_1\ (\texttt{else}\ e_2)^?$
  \\
    && | &
    $\texttt{for:}\ x\ \texttt{:=}\ e_1\ \texttt{to}\ e_2\ \texttt{begin}\ e_3\ \texttt{end}$
  \\
    && | &
    $\texttt{§}C \mid \texttt{‘} C\ \texttt{(} e_1 \texttt{,} \dots \texttt{,} e_n \texttt{)} \mid \texttt{(} e_1 \texttt{,} \dots \texttt{,} e_n \texttt{)} \mid e \texttt{.<} \mathit{proj} \texttt{>}$
  \\
    && | &
    $\texttt{[]} \mid e_1\ \texttt{::}\ e_2$
  \\
    && | &
    $\texttt{‘} C\ \texttt{\{} e_1 \texttt{,} \dots \texttt{,} e_n \texttt{\}} \mid \texttt{\{} e_1 \texttt{,} \dots \texttt{,} e_n \texttt{\}} \mid e \texttt{.\{} \mathit{fld} \texttt{\}} \mid e_1\ \texttt{<-\{} \mathit{fld} \texttt{\}}\ e_2$
  \\
    && | &
    $\texttt{ref}\ e \mid \texttt{!} e \mid e_1\ \texttt{<-}\ e_2$
  \\
    && | &
    $\texttt{match:}\ e_0\ \texttt{with}\ \mathit{br}_1 \texttt{|} \dots \texttt{|}\ \mathit{br}_n\ (\texttt{|\_}\ (\texttt{as}\ s)^?\ \texttt{=>}\ e)^?\ \texttt{end}$
  \\
    && | &
    $e \texttt{.[} \mathit{fld} \texttt{]} \mid \texttt{Xchg}\ e_1\ e_2 \mid \texttt{CAS}\ e_1\ e_2\ e_3 \mid \texttt{FAA}\ e_1\ e_2$
  \\
    && | &
    $\texttt{Proph} \mid \texttt{Resolve}\ e_0\ e_1\ e_2$
  \\
    branch &
    $\mathit{br}$
    & $\Coloneqq$ &
    $C\ (x_1 \dots x_n)^?\ (\texttt{as}\ s)^?\ \texttt{=>}\ e$
  \\
    && | &
    $\texttt{[]}\ (\texttt{as}\ s)^?\ \texttt{=>}\ e \mid x_1\ \texttt{::}\ x_2\ (\texttt{as}\ s)^?\ \texttt{=>}\ e$
  \\
    toplevel value &
    $v$
    & $\Coloneqq$ &
    $t \mid \texttt{\#} n \mid \texttt{\#} b$
  \\
    && | &
    $\texttt{fun:}\ x_1 \dots x_n\ \texttt{=>}\ e \mid \texttt{rec:}\ f\ x_1 \dots x_n\ \texttt{=>}\ e$
  \\
    && | &
    $\texttt{§}C \mid \texttt{‘} C\ \texttt{(} v_1 \texttt{,} \dots \texttt{,} v_n \texttt{)} \mid \texttt{(} v_1 \texttt{,} \dots \texttt{,} v_n \texttt{)}$
  \\
    && | &
    $\texttt{[]} \mid v_1\ \texttt{::}\ v_2$
\end{tabular}
\caption{\ZooLang syntax (omitting mutually recursive toplevel functions)}
\label{fig:zoo}
\end{figure}


The \ZooLang syntax is given in \cref{fig:zoo}\footnote{More precisely, it is the syntax of the surface language, including \Rocq notations.}, omitting mutually recursive toplevel functions that are treated specially.
Expressions include standard constructs like booleans, integers, anonymous functions (that may be recursive), applications, \ocamlinline{let} bindings, sequence, unary and binary operators, conditionals, \ocamlinline{for} loops, tuples.
In any expression, one can refer to a \Rocq term representing a \ZooLang value (of type \coqinline{val}) using its \Rocq identifier.
\ZooLang is deeply embedded: variables (bound by functions and \ocamlinline{let}) are quoted as strings.

Data constructors (immutable memory blocks) are supported: $\texttt{§}C$ represents a constant constructor (\eg $\texttt{§}\texttt{None}$), and $\texttt{‘} C\ \texttt{(} e_1 \texttt{,} \dots \texttt{,} e_n \texttt{)}$ represents a non-constant constructor (\eg $\texttt{‘} \texttt{Some( } e \texttt{ )}$).
Projections of the form $e \texttt{.<} \mathit{proj} \texttt{>}$ (including on tuples: $\texttt{(} x, y \texttt{).<1>}$) can be used to obtain a specific component of a tuple or data constructor.
\ZooLang supports shallow pattern matching (patterns cannot be nested) on data constructors with an optional fallback case.

Mutable memory blocks are constructed using either the untagged record syntax $\texttt{\{} e_1 \texttt{,} \dots \texttt{,} e_n \texttt{\}}$ or the tagged record syntax $\texttt{‘} C\ \texttt{\{} e_1 \texttt{,} \dots \texttt{,} e_n \texttt{\}}$.
Reading a record field can be performed using $e \texttt{.\{} \mathit{fld} \texttt{\}}$ and writing to a record field using $e_1\ \texttt{<-\{} \mathit{fld} \texttt{\}}\ e_2$.
Pattern matching can also be used on mutable tagged blocks provided that cases do not bind anything --- in other words, only the tag is examined, no memory access is performed.
Mutable references are supported: $\texttt{ref}\ e$ creates a reference, $\texttt{!} e$ reads a reference and $e_1\ \texttt{<-}\ e_2$ writes into a reference.
There is no built-in syntax for arrays; they are supported through the \ocamlinline{Array} standard library module (\eg \texttt{array\_make}).

Note that \ZooLang follows \OCaml in sometimes eschewing orthogonality to provide more compact memory representations: constructors are $n$-ary instead of taking a tuple as parameter, and the tagged record syntax is distinct from a constructor taking a mutable record as parameter. In each case the simplifying encoding would introduce an extra indirection in memory, which is absent from the \ZooLang semantics. Performance-conscious experts care about these representation choices, and we care about faithfully modeling their programs.

Parallelism is mainly supported through the \ocamlinline{Domain} standard library module, including domain-local storage. Atomic operations are provided as built-in constructs (\texttt{Xchg}, \texttt{CAS}, \texttt{FAA}; see \cref{sec:atomic}).

The \texttt{Proph} and \texttt{Resolve} constructs model \emph{prophecy variables}~\citep{DBLP:journals/pacmpl/JungLPRTDJ20}, see \cref{sec:prophecy}.

\subsection{Specifications and proofs}
\label{subsec:stack-specs-and-proofs}

Once the translation to \ZooLang is done, the user can write specifications and prove them in \Iris.
For instance, the specification of the \ocamlinline{stack_push} function from \cref{fig:stack} could be:

\begin{coqcode}
Lemma stack_push_spec t ι v :
  <<< stack_inv t ι
    | ∀∀ vs, stack_model t vs >>>
    stack_push t v @ ↑ι
  <<< stack_model t (v :: vs)
    | RET (); True >>>.
\end{coqcode}

It uses a \emph{logically atomic specification}~\citep{DBLP:conf/ecoop/PintoDG14}, which has been proven~\citep{DBLP:journals/pacmpl/BirkedalDGJST21} to be equivalent to \emph{linearizability}~\citep{DBLP:journals/toplas/HerlihyW90} in sequentially consistent memory models.

Similarly to \href{https://en.wikipedia.org/wiki/Hoare_logic}{Hoare triples},
the specification is formed of a precondition and a postcondition, represented in angle brackets.
But each is split in two parts, a \emph{public} or \emph{atomic} condition, and a \emph{private} condition.
Following standard \Iris notations, the private conditions are on the outside (first line of the precondition, last line of the postcondition) and the atomic conditions are inside.

For this particular operation, the private postcondition is trivial.
The private precondition $\mathtt{stack\_inv}\ t$ is the stack invariant.
Intuitively, it asserts that $t$ is a valid concurrent stack.
More precisely, it defines a concurrent protocol that $t$ must respect at all times.

The atomic pre- and post-conditions specify the linearization point of the operation: during the execution of \texttt{stack\_push}, the abstract state of the stack held by $\mathtt{stack\_model}$ is atomically updated from $\mathit{vs}$ to $\mathit{v} :: \mathit{vs}$ when $\mathit{v}$ is atomically pushed at the top of the stack.

%%% Local Variables:
%%% mode: LaTeX
%%% TeX-master: "main"
%%% End:
