\section{Memory safety}

Concurrency creates tensions between performance and memory-safety. The \OCaml maintainers intend to maintain memory-safety for all \OCaml programs, including racy concurrent programs. They have tried to imprint this focus on memory-safety to library authors as well.

Performance-sensitive \OCaml libraries are sometimes written using unsafe primitives, that may break memory-safety if used incorrectly. It is the responsibility of code authors to ensure that the preconditions of those unsafe primitives are satisfied to ensure safety. Unfortunately, the addition to parallel code execution in \OCamlFive broke the safety of some existing code: it adds more possible interleaving and may invalidate safety reasoning.

\subsection{Dynarray} In January 2023, Gabriel Scherer proposed the addition of \ocamlinline{Dynarray}, a module of (sequential) resizable arrays, to the \OCaml standard libary.\footnote{\url{https://github.com/ocaml/ocaml/pull/11882}} 

\paragraph{A concurrent safety problem}

Resizable arrays are implemented as a record with two mutable fields, a \ocamlinline{size} field that stores the current length of the resizable array, and a \ocamlinline{data} field storing the ``backing array'' a (non-resizable) array of size at least \ocamlinline{size} elements. When a user adds a new element to the end of a resizable array, it typically suffices to write the new element at index \ocamlinline{size} in the backing array, and then increment the \ocamlinline{size} field. But when the \ocamlinline{size} field reaches the actual length of the backing array (which we call the ``capacity'' of the resizable array), we first need to allocate a new, larger backing array, to copy the values from the old to the new backing array, and to overwrite the \ocamlinline{data} field with the new backing array.

This operation of adding a new element is performance-critical for resizable arrays. After the size check and potential resizing, sequentially we know that the backing array has enough space, and we can write the new element using an \ocamlinline{unsafe_set} primitive that avoids a redundant bound-check on array access. This optimization can provide speedups of up to 20\% in certain scenarios.

\begin{ocamlcode}
let add_last a x = (* sequential version *)
  let size = a.size in
  if Array.length a.data = size then ensure_capacity a (size + 1);
  a.size <- size + 1;
  Array.unsafe_set a.data size x
\end{ocamlcode}

Unfortunately, the safety reasoning becomes incorrect in presence of parallelism: another thread could mutate the backing array after the size check and before the unsafe write, for example \ocamlinline{Dynarray.reset} which sets the size to \ocamlinline{0} and replaces the backing array by an empty array.
Writing to a (non-resizable) array outside its bounds without bound checks breaks memory safety, so this implementation was memory-safe under \OCamlFour but it becomes memory-unsafe under \OCamlFive. On the other hand, \ocamlinline{Dynarray} is explicitly documented as a sequential data structure, so it is the user responsibility to respect this precondition by using appropriate synchronization (for example a mutex) to prevent data races. Some performance-obsessed users would argue that if \emph{other} users fail in their responsibility of ensuring sequential access, then they do not deserve memory-safety. The \OCaml maintainers and standard library authors consider on the other hand that memory-safety should be preserved even in this case: it can only be lifted for operations that are explicitly marked as unsafe, and hopefully have simple, checkable preconditions. So they decided to change the optimization of \ocamlinline{Dynarray} to guarantee memory-safety even in presence of racy concurrent usage.

The proposed implementation reads the \ocamlinline{data} field to get the backing array $b$ and performs the resizing check. If no resizing is necessary, then an unsafe write is performed as before (on the backing array $b$, without re-reading the \ocamlinline{data} field). In the infrequent case where resizing is necessary, the operation retries afterwards. (A bound-checking write would suffice for safety.)

\begin{ocamlcode}
let rec add_last a x = (* memory-safe under concurrency *)
  let {size; data} = a in
  if Array.length data >= size
  then ( ensure_capacity a (size + 1); add_last a x )
  else ( a.size <- size + 1; Array.unsafe_set arr size x )
\end{ocamlcode}

\paragraph{A concurrent verification problem} We verified (a representative fragment of) the proposed \ocamlinline{Dynarray} implementation in \ZooLang, proving that it is functionally correct under sequential usage, and that it does preserve memory-safety even under concurrent usage. The verified fragment is now part of the \Zoo standard library, and can be used in further verification projects.

Informally, the verification relies on two different invariants:
\begin{itemize}
\item Functional correctness relies on a \emph{strong} invariant,
  which may not be preserved under concurrent usage; for example that
  the content of the \ocamlinline{size} field is smaller than the
  size of the backing array.
\item Memory-safety relies on a \emph{weak} invariant that does not
  suffice to prove correctness, but is preserved under concurrent
  usage. For this \ocamlinline{Dynarray} implementation, the weak
  invariant is that the backing array remains well-typed and that the
  content of the \ocamlinline{size} field is non-negative.
\end{itemize}
To informally check the two desired properties, it suffices to check
that the strong invariant is indeed a sequential invariant, and that
implies functional correctness; and separately that the weak invariant
is preserved by all operations on the data structure.

In our mechanized verification, the strong invariant becomes a \emph{model}
of the data structure, and we verify separation-logic triples where
each operation can assume unique ownership of the model in input, and
has to return ownership of a valid model in output, as is
standard. For example, the ``functional correctness'' lemma for the
\ocamlinline{add_last} function is the following:

\begin{coqcode}
Lemma dynarray_2_push_spec t vs v :
  {{{ dynarray_2_model t vs }}}
    dynarray_2_push t v
  {{{ RET (); dynarray_2_model t (vs ++ [v]) }}}.
\end{coqcode}

This lemma establishes correctness in a purely-sequential usage, but
also under any concurrent usage where correct synchronization is used to
guarantee unique ownership of the model.

The formal counterpart of our weak invariant is a \emph{semantic type}, following the approach of the \RustBelt project~\citep*{DBLP:journals/pacmpl/0002JKD18}, which also deals with unsafe fragments within a language intended to be safe. Note that the \RustBelt semantic types are derived from \Rust types, where mutating operations get a mutable borrow and thus unique ownership (for a time). In contrast, our semantic types carry no ownership of the values they manipulate, so they are stored entirely as invariants, and all interactions with the structure must be shown to preserve this invariant atomically. The invariant must be robust against any interference coming from any other function called in parallel on the same structure.

The definition of the semantic type for dynarrays, and the statement of memory safety (or in fact semantic typing) for \ocamlinline{add_last} look as follows:
\begin{coqcode}
Definition itype_dynarray_2 ty t : iProp Σ :=
  ∃ l,
  ⌜t = #l⌝ ∗
  inv nroot (
    ∃ (sz : nat) cap data,
    l.[size] ↦ #sz ∗
    l.[data] ↦ data ∗ itype_array ty cap data
  ).

Lemma dynarray_2_push_type ty t v :
  {{{ itype_dynarray_2 ty t ∗ ty v }}}
    dynarray_2_push ty t v
  {{{ RET (); True }}}.
\end{coqcode}

The predicate \coqinline{ty} is the semantic type for the elements of the array --- it must hold for each value of the backing array and for the new value \ocamlinline{v}.

\Xgabriel{Je crois avoir compris qu'il y a des choses intéressantes à dire sur le raisonnement sur les itérateurs (\ocamlinline{iteri} par exemple), mais je ne saurais pas quoi écrire à ce sujet.}

\subsection{\Saturn: Single-producer or Single-consumer queues}

Similar problems of memory-safety occur in the concurrent data structures of the \Saturn library. For example, we explained (\cref{par:michael-scott-cleanup}) that the \Saturn implementation of the Michael-Scott MPMC queue is careful to ``erase'' values stored in the queue to avoid memory leaks. This erasure is performed by writing a non-type-safe dummy value, \ocamlinline{Obj.magic ()}. We define a semantic type for the MPMC queue which is essentially its concurrent invariant, and prove that this unsafe idiom does not endanger memory-safety.

On the other hand, the efficient implementations of single-consumer or single-producer queues in \Saturn do \emph{not} respect the general \OCaml recommendation of favoring safety over performance: if a library user uses a single-consumer structure from two consumers in racy ways, they lose memory safety. Our correctness results imply memory-safety when the single-consumer or single-producer protocol is respected; when the caller does own its end of the structure uniquely, as our precondition requires. But we cannot prove memory-safety with only persistent preconditions, it does not hold.

The \Saturn authors also provide ``safe'' variants of their data structures, which are slightly slower but memory-safe even for unintended concurrent usage. This typically adds an indirection, such as using the type \ocamlinline{'a option} instead of \ocamlinline{'a}, which provides a type-safe \ocamlinline{None} value for dummies.

%%% Local Variables:
%%% mode: LaTeX
%%% TeX-master: "main"
%%% End:
