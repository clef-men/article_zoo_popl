\section{Physical equality}
\label{sec:physical_equality}

The notion of \emph{physical equality} is ubiquitous in fine-grained concurrent algorithms.
It appears not only in the semantics of the \ocamlinline{(==)} operator, but also in the semantics of the \ocamlinline{Atomic.compare_and_set} primitive, which atomically sets an atomic reference to a desired value if its current content is physically equal to an expected value.
This primitive is commonly used to try committing an atomic operation in a retry loop, as in the \ocamlinline{push} and \ocamlinline{pop} functions of \cref{fig:stack}.

\subsection{Physical equality in \HeapLang}

In \HeapLang, this primitive is provided but restricted.
Indeed, its semantics is only defined if either the expected or the desired value fits in a single memory word in the \HeapLang value representation: literals (booleans, integers and pointers\footnote{\HeapLang allows arbitrary pointer arithmetic and therefore inner pointers. This is forbidden in both \OCaml and \ZooLang, as any reachable value has to be compatible with the garbage collector.}) and literal injections\footnote{\HeapLang has no primitive notion of constructor, only pairs and injections (left and right).}; otherwise, the program is stuck.
In practice, this restriction forces the programmer to introduce an indirection~\cite{iris-examples,DBLP:journals/pacmpl/JungLPRTDJ20,DBLP:conf/cpp/VindumB21} to physically compare complex values, \eg lists.
Furthermore, when the semantics is defined, values are compared using their \Rocq representations; physical equality boils down to \Rocq equality.

\subsection{Physical equality in \OCaml}

In \OCaml, \emph{physical} equality is tricky.
\emph{Structural} equality \ocamlinline{v1 = v2}, which we describe in \cref{sec:structural_equality}, would be the preferred way of comparing values, and using physical equality \ocamlinline{v1 == v2} is often an unintentional mistake.
However, physical equality is typically much faster than structural equality, as it compiles to only one assembly instruction instead of traversing the value.
Also, the \ocamlinline{Atomic.compare_and_set} requires the comparison to be atomic, ruling out structural equality.

Physical equality is in a counter-intuitive situation: it is very simple to \emph{implement} (in the \OCaml compiler, or in an interpreter, etc.) but difficult to \emph{specify} precisely. To make verification practical, we need a specification at the level of \emph{source} \OCaml (or \Zoo) programs, using a high-level representation of values, as close to the source as reasonably possible, that we call \emph{abstract} values. On the other hand, its implementation typically work with \emph{low-level} values, and its observable behavior depends on compiler transformations that happen in-between the two abstraction levels.
This difficulty can result in dangerous gaps between the programming language used to write code and the semantics used for its verification.

\ZooLang has a grammar of values, and most operations are specified by defining how they compute with \ZooLang values. Its definition may look as follows in \Rocq (simplified):
\begin{coqcode}
Inductive literal :=            Inductive val :=
 | Bool (b : bool)                | Lit (lit : literal)
 | Int (n : nat)                  | Recs (i : nat)
 | Loc (l : location)                    (recs : list (binder * binder * expr))
 | Proph (pid : prophet_id)       | Block (tag : nat) (vs : list val).
 | Poison.
\end{coqcode}

The value $'\texttt{Cons} (42, \texttt{§}\texttt{Nil})$ is represented in \Rocq as \coqinline{Block 1 [Lit (Int 42), Block 0 []]}. Notice that immutable blocks are represented in \Rocq using the \coqinline{Block} constructor directly, and \emph{not} as a location (\coqinline{Loc}) allocated on the heap. We use locations only for \emph{mutable} records. We would say that our representation of \ZooLang values is high-level or \emph{abstract}, as close to the surface syntax as reasonably possible. This distinction is important to make verification pleasant in practice, by reducing the number of locations and heap indirections that the programmer needs to work with during verification. A \ZooLang tuple is directly a tuple, etc., and this design decision of using high-level values is important to the verification experience --- in addition, assuming full owernship of arguments of immutable blocks would be incorrect.

It is tempting to specify, as \HeapLang does, that physical equality decides equality between abstract values. This specification makes sense for immediate values (integers, booleans, constant constructors), and for mutable records which are compared by location. But it is incorrect on immutable blocks, and \HeapLang essentially does not specify its behavior on those values. Yet programmers use physical equality on immutable blocks in practice, as in our example of a Treiber stack of \cref{fig:stack}.

Defining physical equality as \Rocq equality of abstract values is problematic in opposite ways:
\begin{enumerate}
\item Some distinct abstract values are physically equal in OCaml, for example \ocamlinline{0} and \ocamlinline{false}. Their type differ, but it possible to store them in an existential type where they can be compared for physical equality:
\begin{ocamlcode}
type any = Any : 'a -> any
let test1 = Any false == Any 0 (* may return true *)
\end{ocamlcode}
This shows that even on immediate values, specifying physical equality as equality of abstract values is convenient but incorrect in practice.

\item A deeper problem is that some definitionally equal abstract values may be physically distinct.
  Consider for example the case of \emph{immutable blocks} representing constructors and immutable records (as opposed to \emph{mutable blocks} representing mutable records), \eg \ocamlinline{Some 0}.
  The physical comparison of \ocamlinline{Some 0} and \ocamlinline{Some 0} may return either \ocamlinline{true} or \ocamlinline{false}: we cannot determine the result of physical comparison just by looking at the abstract values.
\end{enumerate}

To solve these problems we treat physical equality on abstract values is \emph{non-deterministic} --- even though the comparison instruction that implements it on low-level values is perfectly deterministic.
The question is then: what guarantees do we get when physical equality returns \ocamlinline{true} and when it returns \ocamlinline{false}?
Given such guarantees, denoted by \coqinline{val_physeq} and \coqinline{val_physneq}, the non-deterministic semantics is reflected in the logic through the following specification:

\begin{coqcode}
Lemma physeq_spec v1 v2 :
  {{{ True }}}
    v1 == v2
  {{{ b, RET #b; ⌜(if b then val_physeq else val_physneq) v1 v2⌝ }}}
\end{coqcode}

 The \OCaml manual documents a partial specification for physical equality, which is precise for basic types such as integers or integer references, but does not clearly extend to structured values containing a mix of immutable and mutable constructors, which are present in the programs we verify.
The only guarantee that it provides for all values is: if two values are physically equal, they are also structurally equal. For values that contain immutable constructors, we do not learn anything when they are physically distinct.

We will now explore the specifications of the \ocamlinline{true} and \ocamlinline{false} return cases. We describe our program verification requirements to suggest a precise enough semantics: if it does not say enough, we cannot prove our programs correct. In the other direction we describe some optimizations that \OCaml implementations may perform (gathered through our discussion with \OCaml maintainers), in particular the current compiler, that may rule out certain stronger specifications are incorrect.

\paragraph{Remark} It is tempting to state that physical equality implies equality of \Rocq representations, but incorrect in general as we have seen. Existing work on the modelling of \OCaml physical equality within proof assistants have typically made this simplification, restricting the set of supported values to preserve soundness. We discussed our examples with authors of such earlier formalizations, and this has sometimes uncovered soundness issues on their side, as we discuss in \cref{subsec:related-work-physical-equality}.

\subsection{When physical equality returns \ocamlinline{true}}

Let us go back to the concurrent stack of \cref{fig:stack} and more specifically the \ocamlinline{push} function. Its atomic specification, given in \cref{subsec:stack-specs-and-proofs}, states that if we push a value $v$ onto a stack whose current model is $vs$, then it atomically becomes a stack of model $v :: vs$.
To prove this specification we rely on the fact that, if \ocamlinline{Atomic.compare_and_set} returns \ocamlinline{true}, the current list must be equal to $vs$, in the sense of \Rocq equality for \Zoo values. This equality is strictly stronger than structural equality on mutable types, so we need a more precise specification than provided by the \OCaml manual.

\Zoo supports the following fragment of \OCaml values: booleans, integers, mutable blocks (pointers), immutable blocks, functions.

The easy cases are mutable blocks and functions.
Each of these two classes is disjoint from all others.
We can reasonably assume that, when physical equality returns \ocamlinline{true} and one of the compared values belongs to either of these classes, the two values are actually the same in \Rocq.
As far as we are aware, there is no optimization that could break this.

Booleans, integers and empty immutable blocks (constant constructors) are all represented as immediate integers in \OCaml's low-level representation. This encoding induces conflicts: two abstract values that are distinct in \Rocq may have the same low-level representation.
The semantics of unrestricted physical equality has to reflect these conflicts: on those values that have an immediate representation, our specification does not state that physical equality includes equality of abstract values, it introduces a (simplified) notion of low-level representation and only states that those representations are equal.

Finally, let us consider the case of non-empty immutable blocks.
At runtime, they are represented by pointers to tagged memory blocks.
At first approximation, it is tempting to say that physically equal immutable blocks really are definitionally equal in \Rocq.
Alas, this is not true.
To explain why, we have to recall that the \OCaml compiler may perform \emph{sharing}: immutable blocks containing physically equal fields may be shared.
For example, the following tests may return \ocamlinline{true}:

\begin{ocamlcode}
let test1 = Some 0 == Some 0 (* true *)
let test2 = [0;1]  == [0;1]  (* true *)
\end{ocamlcode}

On its own, sharing is not a problem.
However, coupled with representation conflicts, it can be surprising.
Indeed, consider the \ocamlinline{any} type we introduced previously:

\begin{ocamlcode}
type any = Any : 'a -> any
\end{ocamlcode}

The following tests may return \ocamlinline{true} (they do with \texttt{ocamlopt}, not \texttt{ocamlc}):

\begin{ocamlcode}
let test1 = Any false == Any 0 (* true *)
let test2 = Any None  == Any 0 (* true *)
let test3 = Any []    == Any 0 (* true *)
\end{ocamlcode}

Now, going back to the \ocamlinline{push} function of \cref{fig:stack}, we have a problem.
Given a stack of \ocamlinline{any}, it is possible for the \ocamlinline{Atomic.compare_and_set} to observe a current list (\eg the one-element list \ocamlinline{[Any 0]}) physically equal to the expected list (\eg, \ocamlinline{[Any false]}) while these are actually distinct in \Rocq.
In short, the expected specification of \cref{subsec:stack-specs-and-proofs} is incorrect: we may not get $v :: vs$ back in the model, but a list $v :: vs'$ where $vs'$ is physically equal to $vs$ but not the same abstract value.
To fix this discrepancy, we would need to weaken all our specifications to be formulated \emph{modulo physical equality}, which is non-standard and quite burdensome.

We believe this really is a shortcoming, at least from the verification perspective.
Therefore, we propose to extend \OCaml with \emph{generative immutable blocks}\footnote{\urlAnonymous{https://github.com/clef-men/ocaml/tree/generative_constructors}}.
These generative blocks are just like regular immutable blocks, except they cannot be shared.
Hence, if physical equality on two generative blocks returns \ocamlinline{true}, these blocks are definitionally equal in \Rocq.
At user level, this notion is materialized by \emph{generative constructors}.
For instance, to verify the expected \ocamlinline{push} specification, we can use a generative version of lists:

\begin{ocamlcode}
type 'a glist =
  | Nil
  | Cons of 'a * 'a glist [@generative]
\end{ocamlcode}

We modified the \Zoo translator to support those generative constructors, and modified our implementation of Treiber stack to use the type \ocamlinline{'a glist} instead of \ocamlinline{'a list}, so that we could finally prove the expected, convenient specification. The \ocamlinline{[@generative]} attribute is ignored by the standard \OCaml compiler, and we have an experimental version that disables sharing for generative constructors.

\subsection{When physical equality returns \ocamlinline{false}}

\begin{figure}[tp]
\begin{ocamlcode}
type state =
  | Open of Unix.file_descr
  | Closing of (unit -> unit)

type t =
  { mutable ops: int [@atomic];
    mutable state: state [@atomic]; }

let make fd = { ops = 0; state = Open fd }

let closed = Closing (fun () -> ())
let close t =
  match t.state with
  | Closing _ -> false
  | Open fd as prev ->
      let next = Closing (fun () -> Unix.close fd) in
      if Atomic.Loc.compare_and_set [%atomic.loc t.state] prev next then (
        if t.ops == 0
        && Atomic.Loc.compare_and_set [%atomic.loc t.state] next closed
        then close () ;
        true
      ) else false
\end{ocamlcode}
\caption{\ocamlinline{Rcfd} module from \Eio (excerpt)}
\label{fig:rcfd}
\end{figure}


Most formalizations of physical equality in the literature do not give any guarantee when physical equality returns \ocamlinline{false}.
Many use-cases of physical equality, in particular retry loops, can be verified with only sufficient conditions on \ocamlinline{true}.
However, in some specific cases, more information is needed.

Consider the \ocamlinline{Rcfd} module from the \Eio library, an excerpt of which is given in \cref{fig:rcfd}\footnote{We make use of \emph{atomic record fields} as introduced in \cref{sec:atomic-record-fields}.}.
Thomas Leonard, its author, suggested that we verify this real-life example because of its intricate logical state.
However, we found out that it is also relevant regarding the semantics of physical equality.
Essentially, it consists in wrapping a file descriptor in a thread-safe way using reference-counting.
At creation in the \ocamlinline{make} function, the wrapper starts in the \ocamlinline{Open} state.
At some point, it can switch to the \ocamlinline{Closing} state in the \ocamlinline{close} function, and will remain \ocamlinline{Closing} forever.
Crucially, the \ocamlinline{Open} state changes at most once to \ocamlinline{Closing}, never to another \ocamlinline{Open}.

The interest of \ocamlinline{Rcfd} lies in the \ocamlinline{close} function.
First, the function reads the state.
If this state is \ocamlinline{Closing}, it returns \ocamlinline{false}; the wrapper has been closed.
If this state is \ocamlinline{Open}, it tries to switch to the \ocamlinline{Closing} state using \ocamlinline{Atomic.Loc.compare_and_set}; if this attempt fails, it also returns \ocamlinline{false}.
In this particular case, we would like to prove that the wrapper has been closed, or equivalently that \ocamlinline{Atomic.Loc.compare_and_set} cannot have observed \ocamlinline{Open}.
Intuitively, if we observed a different value then it must be \ocamlinline{Closing}.

Obviously, we need some kind of guarantee related to the \emph{physical identity} of \ocamlinline{Open} when \ocamlinline{Atomic.Loc.compare_and_set} returns \ocamlinline{false}.
If \ocamlinline{Open} were a mutable block, we could argue that this block cannot be physically distinct from itself; no optimization we know of would allow that.
Unfortunately, it is an immutable block, and immutable blocks are subject to more optimizations.
In fact, something surprising but allowed\footnote{\OCaml maintainers developing the \Flambda optimiser have confirmed that it may perform unsharing in certain cases.} by \OCaml can happen: \emph{unsharing}, the dual of sharing.
Indeed, any immutable block can be unshared, that is reallocated as if its definition was inlined.
For example, the following test may theoretically return \ocamlinline{false}:

\begin{ocamlcode}
let x = Some 0
let test = x == x (* false *)
\end{ocamlcode}

Going back to \ocamlinline{Rcfd}, we have a problem: in the second branch, the \ocamlinline{Open} block corresponding to \ocamlinline{prev} could be unshared, which would make \ocamlinline{Atomic.Loc.compare_and_set} fail.
Hence, we cannot prove the expected specification; in fact, the program as it is written has a bug.

To remedy this unfortunate situation, we propose to reuse the notions of generative immutable blocks, that we introduced to prevent sharing, to also forbid unsharing by the \OCaml compiler --- we implemented this in our experiment compiler branch.

Supporting this requires enriching the \Zoo semantics so that each generative block is annotated with a \emph{logical identifier}\footnote{Actually, for practical reasons, we distinguish identified and unidentified generative blocks.} representing its physical identity, much like a pointer for a mutable block.
If physical equality on two generative blocks returns \ocamlinline{false}, the two identifiers are necessarily distinct.
Given this semantics, we can verify the \ocamlinline{close} function.
Indeed, if \ocamlinline{Atomic.Loc.compare_and_set} fails, we now know that the identifiers of the two blocks, if any, are distinct.
As th concurrent protocol has only one \ocamlinline{Open} block whose identifier does not change, it cannot be the case that the current state is \ocamlinline{Open}, hence it is \ocamlinline{Closing}.
We can verify this function after adding the following annotation:

\begin{ocamlcode}
type state =
  | Open of Unix.file_descr [@generative]
  | Closing of (unit -> unit)
\end{ocamlcode}

\subsection{Summary}

In summary, we extended our abstract values with generative immutable blocks, and give the following specification to physical equality in \ZooLang. It can also serves as a precise specification of physical equality of a practical fragment of \OCaml:

\begin{itemize}
\item On values whose low-level representation is
  an immediate integer, physical equality is immediate equality.
\item On mutable blocks at some location, or generative immutable blocks with some identity, physical equality is equality of locations or identities.
\item On immutable blocks, physical-equality is under-specified, but it implies that the blocks have the same tags and their arguments are in turn physically equal.
\item Two values that fall into different categories above are never physically equal.
\end{itemize}

We reflect this specification in the \Zoo program logic, while keeping a reasonably high-level definition of abstract values.

Our verification work uncovered correctness bugs in existing \OCaml concurrent code due to unsharing. We propose to extend the language with a \ocamlinline{[@generative]} annotation for immutable constructors whose physical identity we want to reason about when implementing a fine-grained concurrent data structure. A rule of thumb would be to use it for all non-constant constructors involved in \ocamlinline{Atomic.compare_and_set} operations.

%%% Local Variables:
%%% mode: LaTeX
%%% TeX-master: "main"
%%% End:
