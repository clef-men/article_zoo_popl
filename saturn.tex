\section{\Saturn: A library of standard lock-free data structures}
\label{sec:saturn}

We verified a collection of standard lock-free data structures from the \Saturn, \Eio and \Picos~\citep*{picos} libraries.
It includes stacks, queues (list-based, array-based and stack-based) and bags.
These data structures are meant to be used as is or adapted to fit specific needs.
To cover a wide range of use cases, we provide specialized variants: bounded or unbounded, single-producer (SP) or multi-producer (MP), single-consumer (SC) or multi-consumer (MC).

Due to space constraints, we focus on the most important algorithms and refrain from showing the corresponding (non-trivial) \Iris invariants, which are mechanized in \Rocq.

\subsection{Stacks}

We verified three variants of the Treiber stack~\citep*{thomas1986systems}: 1) unbounded MPMC (the standard one), 2) bounded MPMC, 3) closable unbounded MPMC.
This last variant features a closing mechanism: at some point, some thread can decide to close the stack, retrieving the current content and preventing others from operating on it.
For example, we used it to represent a set of vertex successors in the context of a concurrent graph implementation (not presented in this paper).

As explained in \cref{sec:physical_equality}, the three verified stacks use generative constructors to prevent sharing.
One may ask whether it would be easier to use a mutable version of lists instead.
From the programmer's perspective, this is unsatisfactory because 1) the compiler will typically emit warnings complaining that the mutability is not exploited and 2) it does not really reflect the intent, \ie we want precise guarantees for physical equality, not modify the list.
From the verification perspective, this is also unsatisfactory because the mutable representation is more complex to write and reason about: pointers and points-to assertions versus pure \Rocq list.

Although verified stacks may seem like a not-so-new contribution, it is, as far as we know, the first verification of realistic \OCaml implementations.
For comparison, the exemplary concurrent stacks verified in \Iris~\citep*{iris-examples} all suffer from the same flaw: they need to introduce indirections (pointers) to be able to use the compare-and-set primitive.

\subsection{List-based queues}

We verified four variants of the Michael-Scott queue~\citep*{DBLP:conf/podc/MichaelS96}: unbounded MPMC (the standard one), bounded MPMC, unbounded MPSC and unbounded SPMC.
The SPMC queue is used to implement one of the bags---a relaxed queue guaranteeing only per-producer ordering.

In the \Iris literature, \citet*{DBLP:conf/cpp/VindumB21} established contextual refinement of the Michael-Scott queue while \citet*{DBLP:journals/pacmpl/MulderK23} proved logical atomicity.
However, we had to redesign and extend the invariant for several reasons.

\paragraph{Efficient implementation.}\label{par:michael-scott-cleanup}
The Michael-Scott essentially consists of a singly linked list of nodes that only grows over time.
The previously verified implementations, implemented in \HeapLang, use a double indirection to represent the list~\citep*[\figurename~2]{DBLP:conf/cpp/VindumB21}.
Similarly to the Treiber stack, this is made so as to be able to use the compare-and-set primitive of \HeapLang. \citet*{DBLP:conf/cpp/VindumB21} write:
\begin{quotation}
  A node is a pointer to either none or some of a pair of a value and a pointer to the next node.
  The pointer serves to make nodes comparable by pointer equality such that pointers to nodes can be changed with CAS.
\end{quotation}
In \OCaml, this would correspond to introducing extra atomic references (\ocamlinline{Atomic.t}) between the nodes.
Using atomic record fields, we can represent the list more efficiently, without the extra indirection.
However, there is one subtlety: in this new representation, we need to clear the outdated nodes so that their value is no longer reachable and can be garbage-collected, in other words to prevent a memory leak.
This subtlety is not discussed in the original implementation~\citep*{DBLP:conf/podc/MichaelS96} designed for non-garbage-collected languages, but it is folklore; it is implemented in \Saturn (\href{https://github.com/ocaml-multicore/saturn/pull/64}{saturn\#64}) and we verified that it preserves correctness.

To deal with this representation in separation logic, we introduce the notion of \emph{explicit chain} that allows decoupling the chain structure formed by the nodes and the content of the nodes.
Concretely, the assertion $\textlog{xchain}\ \textit{dq}\ \loc{}s\ \textit{dst}$ represents a chain linking locations $\loc{}s$ and ending at value $\textit{dst}$; $\textit{dq}$ is a discardable fraction~\citep*{DBLP:conf/cpp/VindumB21} that controls the ownership of the chain.
This notion is very flexible as it is independent of the rest of the structure.
As a matter of fact, we used it and its generalization to doubly linked list more broadly, to verify other algorithms.
All the variants of Michael-Scott we verified rely on it.
In particular, it was quite straightforward to extend the invariant of the bounded queue, where nodes carry more (mutable and immutable) information.

\paragraph{External linearization point.}
Our work also revealed another interesting aspect that is not addressed in the literature, as far as we know.
None of the previously verified implementations deal with the \ocamlinline{is_empty} operation, that consists in reading the sentinel node and checking whether it has a successor.
It it has no successor, it is necessarily the last node of the chain, hence the queue is empty.
If it does have a successor, \ocamlinline{is_empty} returns \ocamlinline{false}, meaning we must have observed a non-empty queue.
However, this last part is more tricky than it may seem.
Indeed, it may happen that 1) we read the sentinel while the queue is empty, 2) other operations fill and empty again the queue so that the sentinel is outdated, 3) we read the successor of the former sentinel while the queue is still empty.
The crucial point here is that \ocamlinline{is_empty} is linearized when the first \ocamlinline{push} operation filled the queue.
In other words, the linearization point of \ocamlinline{is_empty} is triggered by another operation; this is called an \emph{external linearization point}.
To handle this in the proof, we introduced a mechanism in the invariant to transfer the \Iris resource materializing the linearization point%
\footnote{
This resource is known as an \emph{atomic update}.
\citet*{DBLP:journals/pacmpl/MulderK23} provide a good description.
}
from \ocamlinline{is_empty} to \ocamlinline{push} and vice versa.

\subsection{Stack-based queues}

A standard way to implement a sequential queue is to use two stacks: producers push onto the \emph{back stack} while consumers pop from the \emph{front stack}, stealing and reversing the back stack when needed.
Based on this simple idea, Vesa Karvonen developed a new lock-free concurrent queue.
We verified the MPMC variant used in \Picos and the closable MPSC variant used in \Picos.

Similarly to the sequential implementation, the two stacks are mainly immutable.
As both stacks are updated using compare-and-set, generative constructors are needed to reason about physical equality.

Similarly again, producers and consumers work concurrently on separate stacks, limiting interference.
The key difference compared to the sequential version is that the algorithm has to deal with the concurrent back stack reversal in a lock-free manner.
Essentially, the concurrent protocol --- and therefore the \Iris invariant --- includes a \emph{destabilization} phase during which a new back stack pointing to the former one awaits to be \emph{stabilized}, which happens when the reversed former back stack becomes the new front stack.
In practice, the synchronization is a bit tricky and relies on the indices of the elements.

%%% Local Variables:
%%% mode: LaTeX
%%% TeX-master: "main"
%%% End:
