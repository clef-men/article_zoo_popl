\section{\ocamlinline{Rcfd}: Parallelism-safe file descriptor}
\label{sec:rcfd}

\newcommand{\inv}{%
  \textcolor{blue}{\textlog{inv}}%
}
\newcommand{\closing}{%
  \textcolor{red}{\textlog{closing}}%
}

\begin{figure}[tp]
  \begin{mathpar}
      \iSpec{
        \iPredTwo\ 1
      }{
        \texttt{make}\ \textit{fd}
      }{
        t
      }{
        \inv\ t\ \textit{fd}\ \iPredTwo
      }
    \and
      \iSpec{
          \inv\ t\ \textit{fd}\ \iPredTwo
        \\
          \iWp{
            \textit{closed}\ \texttt{()}
          }{
            \iPredThree\ \texttt{false}
          }
        \\
          \forall q \ldotp
          \iPredTwo\ q \iWand
          \iWp[
            \textit{res}
          ]{
            \textit{open}\ \textit{fd}
          }{
            \iPredTwo\ q \iSep
            \iPredThree\ \texttt{true}\ \textit{res}
          }
      }{
        \texttt{use}\ t\ \textit{closed}\ \textit{open}
      }{
        \textit{res}
      }{
        \exists b \ldotp
        \iPredThree\ b\ \textit{res}
      }
    \and
      \iSpec{
          \inv\ t\ \textit{fd}\ \iPredTwo
        \\
          \iPredTwo\ 1 \iWand
          \exists \textit{chars} \ldotp
          \textlog{unix-fd}\ \textit{fd}\ 1\ \textit{chars}
      }{
        \texttt{close}\ t
      }{
        \texttt{()}
      }{
        \closing\ t
      }
    \and
      \iSpec{
        \inv\ t\ \textit{fd}\ \iPredTwo
      }{
        \texttt{remove}\ t
      }{
        o
      }{
          \closing\ t
        \\
          o = \textlog{None} \iOr o = \textlog{Some}\ \textit{fd} \iSep \iPredTwo\ 1
      }
  \end{mathpar}
  \caption{\texttt{Rcfd} specification (excerpt)}
  \label{fig:rcfd_spec}
\end{figure}


As mentioned in \cref{sec:physical_equality}, the \ocamlinline{Rcfd} module from the \Eio library is particularly interesting in several respects.
Not only does it justify the introduction of generative constructors in \OCaml, but it also demonstrates the use of \Iris for expressing realistic concurrent protocols.

\paragraph{Specification.}
The \ocamlinline{Rcfd} module provides a parallelism-safe file descriptor (FD) relying internally on reference-counting.
Interestingly, it is used in \Eio in two different ways, more precisely two different ownership regimes: 1) any thread can try to access or close the FD; 2) any thread can try to access the FD but only the owner thread can close it --- and is responsible for closing it.
To verify all uses, the specification of \ocamlinline{Rcfd} has to support both ownership regimes.
However, due to space constraints, we consider a simplified specification given in \cref{fig:rcfd_spec}.
The full verified specification can be found in the mechanization.
%
The specification features four operations%
\footnote{
We omitted two non-essential operations: \ocamlinline{is_open} and \ocamlinline{peek}.
}:

\ocamlinline{make} creates a new object $t$ of type \ocamlinline{Rcfd.t} wrapping a given FD $\mathit{fd}$, yielding the (persistent) invariant $\inv\ t\ \mathit{fd}\ \iPredTwo$, where $\iPredTwo$ is an arbitrary fractional predicate.
Crucially, the user must provide the full predicate $\iPredTwo\ 1$, which is stored in the invariant.
Once it is created, a wrapped FD can be accessed through the \ocamlinline{use} operation and closed through the \ocamlinline{close} operation.

\ocamlinline{use} requires the invariant along with the weakest preconditions of the $\mathit{closed}$ function, that is called if the FD has been flagged as closed, and $\mathit{open}$ function, that is called if the FD is still open.
To control the postconditions and the weakest preconditions, the user can choose an arbitrary predicate $\iPredThree$ parameterized by a boolean indicating whether the $\mathit{closed}$ (\ocamlinline{false}) or the $\mathit{open}$ (\ocamlinline{true}) was called.
The $\mathit{open}$ function is given a fraction of $\iPredTwo$, thereby accessing the FD.

\ocamlinline{close} requires the invariant and proving that the full ownership of $\iPredTwo$ entails the full ownership of the FD $\mathit{fd}$, which is necessary to call \ocamlinline{Unix.close}.
It yields $\closing\ t$, a persistent resource witnessing that $t$ has been flagged as closed.
Actually, the wrapped FD is not closed immediately.
It will be closed only once it is possible, meaning all ongoing calls to \ocamlinline{use} owning a fraction of the FD end.

Alternatively, instead of closing the FD, \ocamlinline{remove} tries to retrieve the full ownership of $\iPredTwo$.
To achieve it, it exploits the same mechanism as \ocamlinline{close} --- flagging $t$ as closed as witnessed by $\closing\ t$ --- but also waits until all \ocamlinline{use} calls are done.

\paragraph{Logical state.}
Thomas Leonard, the author of \ocamlinline{Rcfd}, suggested verifying it to make sure the informal concurrent protocol he described in the \OCaml interface was correct.
This protocol introduces a notion of monotonic logical state --- modeled in \Iris using a specific resource algebra~\cite{DBLP:conf/cpp/TimanyB21} --- to describe the evolution of a FD.
Originally, there were four logical states but we found that only three are necessary for the verification: \textbf{open}, \textbf{closing/users} and \textbf{closing/no-users}.

In the \textbf{open} state, the FD is available for use, meaning any thread can access it through \ocamlinline{use}.
Physically, this corresponds to the \ocamlinline{Open} constructor.

When some thread flags the FD as closed through \ocamlinline{close} or \ocamlinline{remove}, the state transitions from \textbf{open} to \textbf{closing/users}.
Crucially, there can only be one such thread.
In this state, the FD is not really closed yet because of ongoing \ocamlinline{use} operations.
Physically, this logical transition corresponds to switching from the \ocamlinline{Open} to the \ocamlinline{Closing} constructor using \ocamlinline{Atomic.Loc.compare_and_set}.

Once all \ocamlinline{use} operations have finished, when the reference-count reaches zero, it is time to actually `close' the FD by calling the function carried by the \ocamlinline{Closing} constructor.
This has to be done only once.
The `closing' thread is the one that succeeds in updating the \ocamlinline{Closing} constructor (to a new one carrying a no-op function) using \ocamlinline{Atomic.Loc.compare_and_set}.
At this point, the state transitions from \textbf{closing/users} to \textbf{closing/no-users} and the wrapper no longer owns the FD.

\paragraph{Generative contructors.}
As explained in \cref{sec:physical_equality}, the \ocamlinline{Open} constructor has to be generative to prevent \emph{unsharing}.
In fact, the \ocamlinline{Closing} constructor also has to be generative to prevent \emph{sharing}, otherwise two calls could have a shared value of \ocamline{next} and believe they both won the second update.

%%% Local Variables:
%%% mode: LaTeX
%%% TeX-master: "main"
%%% End:
