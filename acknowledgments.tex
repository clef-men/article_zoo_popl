\begin{acks}
  The main contributions of the present work are program
  verification work in \Rocq, and were conducted by Clément
  Allain. The first experiments that ultimately became the \Zoo
  project started in September 2023, as a thin layer on top of
  \HeapLang.\footnote{\url{https://github.com/clef-men/heap_lang}}
  Central aspects of this work came from an Autumn 2023 meeting with
  Thomas Leonard (author of \Eio), Vesa Karvonnen (author of \Kcas,
  \Picos, co-author of \Saturn) and Carine Morel (co-author of
  \Saturn) visiting INRIA Paris. Clément Allain did early verification
  work on \Dynarray, \Rcfd and \Saturn in \HeapLang, and in parallel
  started implementing \Parabs as a simplified scheduler implemented
  in \OCaml.

  The project migrated to a new \Iris language, \ZooLang, in January
  2024, with existing developments migrated and extended with better
  support for \OCaml features, in particular algebraic datatypes and
  pattern matching. A large amount of verification work was done using
  the embedded \ZooLang syntax within \Rocq; Clément Allain
  implemented the \ocamlToZoo translator in February 2025, and
  migrated the developments back to OCaml.

  \Zoo received welcome feedback from Robbert Krebbers and Ralf Jung,
  and we are particularly grateful for the help and advice of Ikke
  Mulder, the author and maintainer of the \Diaframe automation
  framework for \Iris, who helped us improve its integration in \Zoo.

  The modeling choices for the semantics of \OCaml, and physical
  equality in particular, were discussed fruitfully with the Cambium
  team at INRIA Paris, in particular François Pottier. Vincent Laviron
  gave excellent feedback as an expert in the semantics assumed by
  \OCaml optimizers, and in particular pointed at the problem of
  unsharing when reasoning about physical equality.

  The side-effect contributions of the present work are support for
  atomic record fields in \OCamlFive, motivated by our study of
  concurrent code written by experts, in particular Vesa
  Karvonnen. This is joint work of Clément Allain and Gabriel Scherer.
  %
  % Gabriel Scherer wrote and evolved the design proposal for atomic
  % fields in May 2024, Clément Allain and Gabriel Scherer implemented
  % several iterations as experimental compiler versions, and Gabriel
  % Scherer did the social work of gathering opinions and eventually
  % a decision on this new language feature.
  %
  The design RFCs and the compiler implementations received feedback
  from several other people. In particular, Basile Clément initially
  proposed\footnote{\url{https://github.com/ocaml/RFCs/pull/39#issuecomment-2147862938}}
  the idea of packing records and their offsets together in an
  %
  \ocamlinline{[%atomic.loc buf.front]}
  %
  construction at type \ocamlinline{int Atomic.Loc.t}. The initial
  design had a more complex first-class-field-like construction
  % 
  \ocamlinline{[%atomic.field front]}
  %
  at type \ocamlinline{('a bag, int) Atomic.Field.t}. Olivier Nicole
  did a full review of the implementation\footnote{\url{https://github.com/ocaml/RFCs/pull/39#issuecomment-2147862938}}, a substantial amount of work.

  The present article was written jointly by Clément Allain and
  Gabriel Scherer, with thoughtful feedback from anonymous reviewers
  from JFLA'25, ITP'25 and POPL'26, and François Pottier.
\end{acks}

