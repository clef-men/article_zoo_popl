\section{Structural equality}
\label{sec:structural_equality}

Structural equality is also supported.
More precisely, it is not part of the semantics of the language but implemented using low-level primitives%
\footnote{
In \OCaml, these primitives correspond to the unsafe functions \ocamlinline{Obj.is_int}, \ocamlinline{Obj.tag}, \ocamlinline{Obj.size} and \ocamlinline{Obj.field}.
}.
The reason is that it is in fact difficult to specify for arbitrary values.
In general, we have to compare graphs --- which implies structural comparison may diverge.

Accordingly, the specification of $v_1\ \texttt{=}\ v_2$ requires the (partial) ownership of a \emph{memory footprint} corresponding to the union of the two compared graphs, giving the permission to traverse them safely.
If it terminates, the comparison decides whether the two graphs are bisimilar (modulo representation conflicts, as described in \cref{sec:physical_equality}).
In \Iris, this gives:

\begin{coqcode}
Lemma structeq_spec v1 v2 footprint,
  val_traversable footprint v1 →
  val_traversable footprint v2 →
  {{{ structeq_footprint footprint }}}
    v1 = v2
  {{{ b, RET #b;
      structeq_footprint footprint ∗
      ⌜(if b then val_structeq else val_structneq) footprint v1 v2⌝ }}}.
\end{coqcode}

Obviously, this general specification is not very convenient to work with.
Fortunately, for abstract values (without any mutable part), we can prove a much simpler variant saying that structural equality boils down to physical equality:

\begin{coqcode}
Lemma structeq_spec_abstract v1 v2 :
  val_abstract v1 →
  val_abstract v2 →
  {{{ True }}}
    v1 = v2
  {{{ b, RET #b; ⌜(if b then val_physeq else val_physneq) v1 v2⌝ }}}
Proof. ... Qed.
\end{coqcode}

\Xgabriel{Je ne trouve pas du tout naturelle la dernière spécification. Lue naïvement elle suggère que (1) si deux valeurs immutables sont structurellement égales, alors elles sont physiquement égales, et que (2) si elles sont structurellement distinctes alors elles sont physiquement distinctes. La suggestion (1) est incorrecte et la suggestion (2) est trop faible. Ce qui se passe c'est que \texttt{val\_phys{eq,neq}} veulent dire autre chose que l'égalité physique, et sont des conséquences à la fois de l'(in)égalité physique et de l'(in)égalité structurelle. Je pense qu'il vaudrait mieux changer leurs noms.}

%%% Local Variables:
%%% mode: LaTeX
%%% TeX-master: "main"
%%% End:
