\section{Related work}
\label{sec:related}

In general there are two approaches to practical program verification:

\subsection{Non-automated verification}

The verified program is translated, manually or in an automated way, into a representation living inside a proof assistant.
The user has to write specifications and prove them.

The representation may be primitive, like Gallina for \Rocq.
For pure programs, \Xgabriel[this is rather straightforward]{I disagree, I believe that hs-to-coq is scientifically problematic as the translation is unsound for higher-order functions and infinite data.}, \eg in \texttt{hs-to-coq}~\cite{DBLP:conf/cpp/Spector-Zabusky18}.
For imperative programs, this is more challenging.
One solution is to use a monad, \eg in \texttt{coq-of-ocaml}~\cite{coq-of-ocaml}, but it does not support concurrency.

The representation may be embedded, meaning the semantics of the language is formalized in the proof assistant.
This is the path taken by some recent works~\cite{DBLP:books/hal/Chargueraud23, DBLP:journals/pacmpl/GondelmanHPTB23, DBLP:conf/sosp/ChajedTKZ19,osiris} harnessing the power of separation logic.
In particular, \CFML~\cite{DBLP:books/hal/Chargueraud23} and \Osiris~\cite{osiris} target \OCaml.
However, \CFML does not support concurrency and is not based on \Iris.
\Osiris, still under development, is based on \Iris but does not support concurrency.

At the time of writing, \HeapLang is thus the most appropriate tool to verify concurrent \OCaml programs. We discussed limitations of \HeapLang in the introduction, and \ZooLang is our proposal to improve on this. Conversely, one notable limitation of \ZooLang today is its lack of support for \OCaml's relaxed memory model.

\subsection{Semi-automated verification}

In semi-automated verification approaches, the verified program is annotated by the user to guide the verification tool: preconditions, postconditions, invariants, \etc.
Given this input, the verification tool generates proof obligations that are mostly automatically discharged.
One may further distinguish two types of semi-automated systems: \emph{foundational} and \emph{non-foundational}.

In \emph{non-foundational} automated verification, the tool and the external solvers it may rely on are part of the trusted computing base.
It is the most common approach and has been widely applied in the literature~\cite{DBLP:journals/jfp/SwamyCFSBY13, DBLP:series/natosec/0001SS17, DBLP:conf/nfm/JacobsSPVPP11, DBLP:conf/icfem/DenisJM22, DBLP:conf/nfm/AstrauskasBFGMM22, DBLP:conf/esop/FilliatreP13, DBLP:journals/pacmpl/LattuadaHCBSZHPH23, DBLP:journals/pacmpl/PulteMSMSK23}, including to \OCaml by \Cameleer~\cite{DBLP:conf/cav/PereiraR20}, which uses the \Gospel specification language~\cite{DBLP:conf/fm/ChargueraudFLP19} and \WhyThree~\cite{DBLP:conf/esop/FilliatreP13}.

In \emph{foundational} automated verification, the proofs are checked by a proof assistant like \Rocq, meaning the automation does not have to be trusted.
To our knowledge, it has been applied to \C~\cite{DBLP:conf/pldi/SammlerLKMD021} and \Rust~\cite{DBLP:journals/pacmpl/GaherSJKD24}.

\Zoo is a non-automated verification framework---except for our use \Diaframe for local automation of separation logic reasoning. We would be interested in moving towards more automation in the future.

\subsection{Physical equality}
\label{subsec:related-work-physical-equality}

There is some literature in proof-assistant research on reflecting physical equality from the implementation language into the proof assistant, for optimization purposes: for example, exposing \OCaml's physical equality as a predicate in \Rocq lets us implement some memoization and sharing techniques in \Rocq libraries.
%
However, axiomatizing physical equality in the proof assistant is difficult, and can result in inconsistencies.

The earlier discussions of this question that we know come from Jourdan's thesis~\cite{DBLP:phd/hal/Jourdan16} (chapter 9), also presented more succintly in \cite{DBLP:journals/jar/BraibantJM14}.
%
This work introduces the Jourdan condition, that physical equality implies equality of values.
%
\cite{boulme:tel-03356701} extends the treatment of physical equality in \Rocq, integrating it in an ``extraction monad'' to control it more safely.
%
There is also a discussion of similar optimizations in \Lean in \cite{lean-pointer-optimizations}.

The correctness of the axiomatization of physical equality depends on the type of the values being compared: axiomatizations are typically polymorphic on any type \coqinline{A}, but their correctness depends on the specific \coqinline{A} being considered.
%
For example, it is easy to correctly characterize physical on natural numbers, and other non-dependent types arising in \Rocq verification projects.
%
One difficulty in \HeapLang and \ZooLang is that they are untyped languages, their representation of \ocamlinline{0} and \ocamlinline{false} has the same type.
%
But our remark that structural equality (in \OCaml) does not necessarily coincide with definitional equality (in \Rocq) also applies to other \Rocq types: our examples with an existential \ocamlinline{Any} constructor (see~\cref{sec:physical_equality}) can be reproduced with $\Sigma$-types.
