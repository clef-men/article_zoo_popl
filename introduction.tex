\section{Introduction}
\label{sec:introduction}

OCaml 5.0 was released on December 15th 2022, the first version of the
OCaml programming language to support parallel execution of \OCaml
threads by merging the \MulticoreOCaml
runtime~\citep*{multicore-ocaml}. It provided basic support in the
language runtime to start and stop coarse-grained threads (``domains''
in \OCaml parlance) and support for strongly sequential atomic
references in the standard library. The third-party library
\texttt{domainslib} offered a simple scheduler for a pool of tasks,
used to benchmark the parallel runtime. A world of parallel and
concurrent software was waiting to be invented.

Shared-memory concurrency is a difficult programming domain, and
existing ecosystems (C++, Java, Haskell, Rust, Go...) took decades to
evolve comprehensive libraries of concurrent abstractions and data
structures. In the last couple years, a handful of contributors to the
OCaml system have been implementing basic libraries for concurrent and
parallel programs in \OCaml, in particular \Saturn~\citep*{saturn},
a library of lock-free thread-safe data structures (stacks, queues,
a work-stealing dequeue, a skip list, a bag and a hash table),
\Eio~\citep*{eio}, a library of asynchronous IO and structured
concurrency, and \Kcas~\citep*{kcas}, a library offering
a software-transactional-memory abstraction for users to build safe
yet efficient thread-safe data structures.

Concurrent algorithms and data structures are extremely difficult to
reason about. Their implementations tend to be fairly short, a few
dozens of lines. There is only a handful of experts able to write such
code, and many potential users. They are difficult to test
comprehensively. These characteristics make them ideally suited for
mechanized program verification.

We embarked on a mission to mechanize correctness proofs of \OCaml
concurrent algorithms and data structures as they are being written,
in contact with their authors, rather than years later. In the
process, we not only gained confidence in these complex new building
blocks, but we also improved the \OCaml language and its verification
ecosystem.

\paragraph{Verification tools for concurrent programs.}

\paragraph{\OCaml language features} When studying the new codebases
of concurrent and parallel data structures, we found a variety of
unsafe idioms, working around expressivity or performance limitations
with the \OCaml language support for lock-free concurrent data
structures. In particular, the support for \emph{atomic references} in
the \OCaml library proved inadequate, as idiomatic concurrent
data-structures need the more expressive feature of \emph{atomic
  record fields}. We designed an extension of \OCaml with atomic
record fields, implemented it as a an experimental compiler variant,
and succeeded in getting it integrated in the upstream \OCaml
compiler: it should be available as part of OCaml 5.4, which is not
yet released at the time of writing.

\paragraph{Specified \OCaml semantics}
