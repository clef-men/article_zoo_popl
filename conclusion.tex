\section{Conclusion and future work}

We presented \Zoo, a framework for verification of concurrent \OCaml~5 programs.
While it is not yet available on \texttt{opam}, it can be installed and used in other \Rocq projects.
%
\Zoo has been used to verify sequential imperative algorithms and a significant library of lock-free data structures.
Its main weakness so far is its memory model, which is sequentially consistent as opposed to the relaxed \OCaml~5 memory model.
In the future we also plan to add exceptions and algebraic effects.

Another interesting direction would be to combine \Zoo with semi-automated techniques.
Similarly to \WhyThree, the simple parts of the verification effort would be done in a semi-automated way, while the most difficult parts would be conducted in \Rocq.

%%% Local Variables:
%%% mode: LaTeX
%%% TeX-master: "main"
%%% End:
