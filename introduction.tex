\section{Introduction}
\label{sec:introduction}

OCaml 5.0 was released on December 15th 2022, the first version of the
OCaml programming language to support parallel execution of \OCaml
threads by merging the \MulticoreOCaml
runtime~\citep*{multicore-ocaml}. It provided basic support in the
language runtime to start and stop coarse-grained threads called
``domains'', and support for strongly sequential atomic references in
the standard library. The third-party library \texttt{domainslib}
offered a simple scheduler for a pool of tasks, used to benchmark the
parallel runtime. A world of parallel and concurrent software was
waiting to be invented.

Shared-memory concurrency is a difficult programming domain, and
existing ecosystems (C++, Java, Haskell, Rust, Go...) took decades to
evolve comprehensive libraries of concurrent abstractions and data
structures. In the last couple years, a handful of contributors to the
OCaml ecosystem have been implementing basic libraries for concurrent and
parallel programs in \OCaml, in particular \Saturn~\citep*{saturn},
a library of lock-free thread-safe data structures (stacks, queues,
a work-stealing queue, a skip list, a bag and a hash table),
\Eio~\citep*{eio}, a library of asynchronous IO and structured
concurrency, and \Kcas~\citep*{kcas}, a library offering
a software-transactional-memory abstraction.

Concurrent algorithms and data structures are extremely difficult to
reason about. Their implementations tend to be fairly short, a few
dozens of lines. There is only a handful of experts able to write such
code, and many potential users. They are difficult to test
comprehensively. These characteristics make them ideally suited for
mechanized program verification.

We embarked on a mission to mechanize correctness proofs of \OCaml
concurrent algorithms and data structures as they are being written,
in contact with their authors, rather than years later. In the
process, we not only gained confidence in these complex new building
blocks, but we also improved the \OCaml language and its verification
ecosystem.

\paragraph{\OCaml language features} When studying the new codebases
of concurrent and parallel data structures, we found a variety of
unsafe idioms, working around expressivity or performance limitations
with the \OCaml language support for lock-free concurrent data
structures. In particular, the support for \emph{atomic references} in
the \OCaml library proved inadequate, as idiomatic concurrent
data-structures need the more expressive feature of \emph{atomic
  record fields}. We designed an extension of \OCaml with atomic
record fields, implemented it as a an experimental compiler variant,
and succeeded in getting it integrated in the upstream \OCaml
compiler: it should be available as part of OCaml 5.4, which is not
yet released at the time of writing.

\paragraph{Verification tools for concurrent programs.} The state-of-the-art approach for mechanized verification of fine-grained concurrent algorithms is \Iris~\citep*{DBLP:journals/jfp/JungKJBBD18}, a mechanized \emph{higher-order} concurrent separation logic with \emph{user-defined ghost state}. Its expressivity allows to precisely capture delicate invariants, and to reason about the linearization points of fine-grained concurrent algorithms, including external~\citep*{DBLP:conf/cpp/VindumFB22} and future-dependent~\citep*{DBLP:journals/pacmpl/JungLPRTDJ20,DBLP:conf/cpp/VindumB21,DBLP:conf/osdi/Chang0STKZ23} linearization. \Iris provides a generic mechanism to define programming languages and program logics for them. Most existing \Iris concurrent verification work has been performed in \HeapLang, the exemplar \Iris language; a concurrent, imperative, untyped, call-by-value functional language.

To the best of our knowledge, it is currently the closest language to \OCamlFive in the \Iris ecosystem --- we review the existing frameworks in \cref{sec:related}. We started our verification effort in \HeapLang, but it eventually proved impractical to verify realistic \OCaml libraries.
Indeed, it lacks basic abstractions such as algebraic data types (tuples, mutable and immutable records, variants) and mutually recursive functions.
Verifying \OCaml programs in \HeapLang requires difficult translation choices and introduces various encodings, to the point that the relation between the source and verified programs can become difficult to maintain and reason about.
It also has very few standard data structures that can be directly reused.
These limitations are well-known in the \Iris community.

We created a new \Iris language, \ZooLang, that can better express concurrent \OCaml programs. Its feature set grew over time as we applied it to more verification scenarios, and we now believe that it allows practical verification of fine-grained concurrent \OCamlFive programs --- including the use of our atomic record fields which were co-designed with \ZooLang.
We were influenced by the \Perennial framework~\citep*{DBLP:conf/sosp/ChajedTKZ19}, which achieved similar goals for the \Go language with a focus on crash-safety.
As in \Perennial, we also provide a translator from (a subset of) \OCaml to \ZooLang: \texttt{ocaml2zoo}. We start from \OCaml code and call our translator to obtain a deep \ZooLang embedding inside \Rocq; we can use lightweight annotations to guide the translation. Inside \Rocq we define specifications using \Iris, and prove them correct with respect to the \ZooLang version, which is syntactically very close to the original \OCaml source.
We call the resulting framework \Zoo.

One notable current limitation of \ZooLang is that it assumes a sequentially-consistent memory model, whereas \OCaml offers a weaker memory model~\citep*{ocaml-memory-model}. We wanted to ensure that we supported practical verification in a sequentially-consistent setting first; in the future we plan to adopt the \OCaml memory model as formalized in \Cosmo~\citep*{DBLP:journals/pacmpl/MevelJP20}. We discuss the impact of this difference in \cref{subsec:memory-model}.

% The formal verification of fine-grained concurrent algorithms involves finding and reasoning about non-trivial linearization points~\cite{DBLP:journals/csur/DongolD15,DBLP:journals/pacmpl/JungLPRTDJ20,DBLP:conf/cpp/VindumB21,DBLP:conf/cpp/VindumFB22,DBLP:conf/osdi/Chang0STKZ23}. In recent years, concurrent separation logic~\cite{DBLP:journals/siglog/BrookesO16} has enabled significant progress in this area.
% In particular, the development of \Iris~\cite{DBLP:journals/jfp/JungKJBBD18}, a state-of-the-art mechanized \emph{higher-order} concurrent separation logic with \emph{user-defined ghost state}, has nourished a rich and successful line of works~\cite{DBLP:journals/pacmpl/JungLPRTDJ20,DBLP:conf/cpp/VindumB21,DBLP:conf/cpp/VindumFB22,DBLP:conf/osdi/Chang0STKZ23,DBLP:conf/cpp/CarbonneauxZKON22,DBLP:journals/pacmpl/JungLCKPK23,DBLP:journals/pacmpl/SomersK24,DBLP:journals/pacmpl/MevelJP20,DBLP:journals/pacmpl/MevelJ21,DBLP:conf/pldi/DangJCNMKD22,DBLP:journals/pacmpl/ParkKMJLKK24,DBLP:conf/pldi/MulderKG22,DBLP:journals/pacmpl/MulderK23}, dealing with external~\cite{DBLP:conf/cpp/VindumFB22} and future-dependent~\cite{DBLP:journals/pacmpl/JungLPRTDJ20,DBLP:conf/cpp/VindumB21,DBLP:conf/osdi/Chang0STKZ23} linearization points, relaxed memory~\cite{DBLP:journals/pacmpl/MevelJP20,DBLP:journals/pacmpl/MevelJ21,DBLP:conf/pldi/DangJCNMKD22,DBLP:journals/pacmpl/ParkKMJLKK24} and automation~\cite{DBLP:conf/pldi/MulderKG22,DBLP:journals/pacmpl/MulderK23}.

\paragraph{Specified \OCaml semantics} Our \Iris mechanization of \ZooLang defines an operational semantics and a corresponding program logic. Our users on the other hand run their program through the standard \OCaml implementation, which is not verified and does not have a precise formal specification. To bridge this formal-informal gap as well as reasonably possible, we carefully audit our \ZooLang semantics to ensure that they coincide with \OCaml's.

In doing so we discovered a hole in state-of-the-art language semantics for program verification (not just for \OCaml), which is the treatment of \emph{physical equality} (pointer quality). Physical equality is typically exposed to language users as an efficient but under-specified equality check, as the physical identity of objects may or may not be preserved by various compiler transformations. It is an essential aspect of concurrent programs, as it underlies the semantics of important atomic instructions such as \texttt{compare\_and\_set}. We found that the current informal semantics in \OCaml is incomplete, it does not allow to reason on programs that use structured data which mix mutable and immutable constructors. Existing formalizations of physical equality in verification frameworks typically restrict it to primitive datatypes, but idiomatic concurrent programs do not fit within this restriction. We propose a precise specification of physical equality in \Zoo that scales to the verification of all the concurrent programs we encountered.

Worse, our discussions with the maintainers of the \OCaml implementation showed that implementors guarantee weaker properties of physical equalities than users assume, in particular they may allow \emph{unsharing}, which makes some existing concurrent programs incorrect. We propose a small new language feature for \OCaml, per-constructor unsharing control, which we also integrate in our \ZooLang translation, to fix affected programs and verify them. Finally, we discussed these subtleties with authors who axiomatize physical equality within Rocq for the purpose of efficient extraction, and we found out that some subtleties we discovered could translate into incorrectness in their axiomatization, requiring careful restrictions.

\paragraph{Verification results} We verified a small library for \ZooLang, typically a subset of the \OCaml standard library. It can serve as building blocks to define our concurrent data structures. (The lack of such a reusable standard library is a current limitation of \HeapLang.) We verified a specific component of the \Eio~library, whose author Thomas Leonard had pointed to us as being delicate to reason about and worth mechanizing. Finally, we verified a large subset of the \Saturn library. Several of these data structures contained verification challenges, which we will describe in the relevant section. The main \Saturn concurrent structures remaining unverified are a skip-list and a hashtable; we have verified its work-stealing queue~\citep*{DBLP:conf/spaa/ChaseL05}, but do not discuss it here for space reasons.

\paragraph{Contributions} In summary, we claim the following contributions:
\begin{enumerate}
\item \Zoo, a practical program verification framework aimed at
  concurrent \OCaml programs, mechanized in Rocq. The language
  \ZooLang comes with a program logic expressed in the \Iris
  concurrent separation logic. A translator \texttt{ocaml2zoo}
  generates Rocq embeddings from source \OCaml programs, and works
  well with \OCaml tooling (\texttt{dune} support).

\item The verification (in a sequentially-consistent model) of
  important structures coming from \Saturn, the \OCamlFive library of
  lock-free data structures. Our implementations and invariants
  sometimes improve over the \Iris state of the art for those data
  structures.

\item The extension of \OCaml with atomic record fields, which after
  significant design, implementation and discussion work have now been
  integrated into upstream \OCaml.

\item The identification of blind spots in existing specifications of
  \emph{physical equality}, and a new specification precise
  enough to reason about \texttt{compare-and-set} in concurrent
  programs.

  In the process we identified a potential bug in existing \OCaml
  programs related to \emph{unsharing}, and we propose a small
  language extension to let users selectively disable unsharing.
\end{enumerate}

\paragraph{Artifact} We include an anonymous artifact containing (1)
our experimental fork of the OCaml compiler implementing atomic arrays
(\cref{subsec:atomic-arrays}) and generative constructors
(\cref{subsec:generative-constructors}), (2) the source of the
\texttt{ocaml2zoo} translator, and (3) the \Zoo repository, which
includes \ZooLang and its metatheory (\texttt{theories/zoo}), verified
OCaml source (in \texttt{lib/}) and corresponding proofs
(in \texttt{theories/}). In particular, the output of
\texttt{ocaml2zoo} is included, for example \text{mpmc\_stack\_1.ml}
in \texttt{lib/zoo\_saturn/} is translated into
\texttt{mpmc\_stack\_1\_\_code.v} in \texttt{theories/zoo\_saturn/}.

%%% Local Variables:
%%% mode: LaTeX
%%% TeX-master: "main"
%%% End:
