\section{Standard data structures}
\label{sec:std}

To save users from reinventing the wheel, we provide a library of verified standard data structures --- more or less a subset of the \OCaml standard library.
Most of these data structures%
\footnote{
For practical reasons, to make them completely opaque, we chose to axiomatize a few functions from the \ocamlinline{Domain} and \ocamlinline{Random} modules.
They could trivially be realized in \Zoo.
}
are completely reimplemented in \Zoo and axiom-free, including the \ocamlinline{Array}%
\footnote{
Our implementation of the \ocamlinline{Array} module is compatible with the standard one.
In particular, it uses the same low-level value representation.
}
module.

\paragraph{Sequential data structures.}

We provide verified implementations of various sequential data structures: array, dynamic array (vector), list, stack, queue (bounded and unbounded), double-ended queue.
We claim that the proven specifications are modular and practical.
In fact, most of these data structures have already been used to verify more complex ones --- we present some in \cref{sec:persistent} and \cref{sec:saturn}.
Especially, we developed an extensive collection of flexible specifications for the iterators of the \ocamlinline{Array} and \ocamlinline{List} modules.
Remarkably, our formalization of \ocamlinline{Array} features different (fractional) predicates to express the ownership of either an entire array, a slice or even a circular slice --- we use it to verify algorithms involving circular arrays, \eg Chase-Lev working-stealing queue~\citep*{DBLP:conf/spaa/ChaseL05} as presented in \cref{sec:ws_queue}.

\paragraph{Concurrent data structures.}

We provide verified implementations of various concurrent data structures: domain%
\footnote{
Domains are the units of parallelism in \OCamlFive.
}
(including domain-local storage), mutex, semaphore, condition variable, write-once variable (also known as \emph{ivar}), atomic array.
Note that there is currently no \ocamlinline{Atomic_array} module in the \OCaml standard library, but we are planning to propose it.
